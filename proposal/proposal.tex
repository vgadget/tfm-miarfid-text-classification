\documentclass[12pt]{article}
\usepackage{lingmacros}
\usepackage{tree-dvips}
\usepackage{multirow}
\usepackage[section]{placeins}
\usepackage[table,xcdraw]{xcolor}
\usepackage[a4paper, total={6in, 8in}]{geometry}
\pagenumbering{gobble}

\begin{document}

    \begin{center}
         \textbf{\huge Master thesis project proposal} \\        
        \vspace{0.25cm}
        Polytechnic University of Valencia\\
        \vspace{0.25cm}
        Adrián Vázquez Barrera \\
        \vspace{0.25cm}
        \textbf{April 2022} 
    \end{center}

    \vspace{0.5cm}
    ~\\
    \textbf{\large Title / Título / Títol}

    \begin{table}[hbt!]
        \centering
        \begin{tabular}{|
        >{\columncolor[HTML]{ECECEC}}l |}
        \hline
        English  \\ \hline
        \end{tabular}
    \end{table}

    Neural news classifier from pre-trained models

    \begin{table}[hbt!]
        \centering
        \begin{tabular}{|
        >{\columncolor[HTML]{ECECEC}}l |}
        \hline
        Español  \\ \hline
        \end{tabular}
    \end{table}

    Clasificador neuronal de noticias a partir de modelos pre-entrenados

    \begin{table}[hbt!]
        \centering
        \begin{tabular}{|
        >{\columncolor[HTML]{ECECEC}}l |}
        \hline
        Valencià  \\ \hline
        \end{tabular}
    \end{table}

    Classificador neuronal de notícies a partir de models pre-entrenats

    ~\\\\
    \textbf{\large Summary / Descripción / Descripció}
    \begin{table}[hbt!]
        \centering
        \begin{tabular}{|
        >{\columncolor[HTML]{ECECEC}}l |}
        \hline
        English  \\ \hline
        \end{tabular}
    \end{table}
    \\It is proposed to study pre-trained linguistic models available in Pytorch, in order to fine-tune them and improve their baseline accuracy. 
    The study will consider metrics such as training cost, accuracy and model performance.
    \\\\The model is intended to be usable by end-users, so the best implementation will be released for production.
    \begin{table}[hbt!]
        \centering
        \begin{tabular}{|
        >{\columncolor[HTML]{ECECEC}}l |}
        \hline
        Español  \\ \hline
        \end{tabular}
    \end{table}
~\\Se propone estudiar los modelos de lenguaje pre-entrenados disponibles en Pytorch, con el fin de afinarlos y mejorar su precisión de partida.  El estudio tendrá en cuenta métricas como el coste del entrenamiento, la precisión y el rendimiento del estos.
\\\\Se pretende que el modelo sea utilizable por los usuarios finales, de modo que la mejor implementación se publicará para su puesta en producción.

\newpage
    \begin{table}[hbt!]
        \centering
        \begin{tabular}{|
        >{\columncolor[HTML]{ECECEC}}l |}
        \hline
        Valencià  \\ \hline
        \end{tabular}
    \end{table}
    ~\\Es proposa estudiar els models de llenguatge pre-entrenats disponibles en Pytorch, amb la finalitat d'afinar-los i millorar la seua precisió de partida. L'estudi tindrà en compte mètriques com el cost de l'entrenament, la precisió i el rendiment d'aquests.
    \\\\Es pretén que el model siga utilitzable pels usuaris finals, de manera que la millor implementació es publicarà per a la seua posada en producció.

    ~\\\\     
    \textbf{\large Keywords / Palabras clave}
    \begin{table}[hbt!]
        \centering
        \begin{tabular}{|
        >{\columncolor[HTML]{ECECEC}}l |}
        \hline
        English  \\ \hline
        \end{tabular}
    \end{table}
    \\
    BERT, Classifier, Computer linguistics, Deployment, Development, Fine-tuning, GPT, Language model, Machine learning, Neural network, News, Pre-trained, PyTorch, Transformers.
    \\
    \begin{table}[hbt!]
        \centering
        \begin{tabular}{|
        >{\columncolor[HTML]{ECECEC}}l |}
        \hline
        Español  \\ \hline
        \end{tabular}
    \end{table}
    \\
    BERT, Clasificador, Lingüística computacional, Despliegue, Desarrollo, Fine-tuning, GPT, Modelo de lenguaje, Aprendizaje automático, Red neuronal, Noticias, Pre-entrenado, PyTorch, Transformers.


\end{document}
